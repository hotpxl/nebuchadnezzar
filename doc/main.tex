\documentclass{ctexart}
\usepackage{hyperref}
\begin{document}
\title{毕业设计说明文档}
\author{李雨田}
\maketitle
\section{源代码}
所有源代码均在 GitHub (https://github.com/hotpxl/nebuchadnezzar) 上。

\section{使用方法和文件结构}

所有源代码和部分数据文件都在代码仓库里。

输入 \texttt{npm install} 以安装所有 Node.js 的依赖。 Python 的依赖需要手动安装。

\subsection{原始数据}

其中 \texttt{scraper} 文件夹下为获取原始论坛数据的脚本。获取之后的数据存储在 \texttt{data} 文件夹下。
目前的版本是基于股吧手机版网页的,效率比从电脑版网页抓取更高,但是数字上有略微的区别。之前是有过电脑版网页抓取
的脚本,后来被覆盖了。在 Git 的历史中可以找到,或者根据手机版修改也不会很麻烦。

股价是通过交易客户端获取的,导出 CSV 格式的文本。通过 \texttt{merger} 中的脚本,和之前的数据整合得到 \texttt{data/merged} 里面的
完整数据。此时已经按具体股票分开了。

其中抓取论坛数据的时候,使用了 Redis 数据库来暂时存储结果。 Redis 的配置文件在 \texttt{database} 文件夹下。

而 \texttt{parser} 文件夹下主要为解析原始数据格式的脚本,在 \texttt{merger} 里会用的。

\subsection{情感数据}

情感数据首先由 \texttt{translator} 文件夹下的脚本,讲之前得到的论坛数据翻译成英文。这里手动注册了若干个百度翻译 API 的帐号轮流使用。
翻译之后的结果由 \texttt{sentiment} 文件夹下的脚本分析情感极性。会调用项目 https://github.com/xiaohan2012/twitter-sent-dnn 的相应代码。
该项目已直接包含在 \texttt{sentiment/twitter} 下。可以先试图将其跑起来,因为会有一些依赖上的问题,也可以参考其 GitHub 上的说明文档。

分析情感的时候,先调用 \texttt{sentiment/create\_task\_queue.py} 生成任务队列存入 Redis 数据库。然后再调用 \texttt{sentiment/extract\_sentiment.py} 从任务队列中
取出任务分析。由于 Python 不支持并行,这里可以手动开启多个 Python 进程以加速任务。

\subsection{数据处理}

此处主要使用 StatsModels 等数学库计算。可以参考 \texttt{thesis\_plots.py} 脚本,运行此脚本可以生成论文中所有的图,包含了论文中所以分析的数学计算。

\end{document}